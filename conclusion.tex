\section{conclusion}
In this study, we assessed the performance of four optimization algorithms—Randomized Hill Climbing (RHC), Simulated Annealing (SA), Genetic Algorithms (GA), and MIMIC—by analyzing their underlying mechanisms and comparing their strengths and weaknesses. These algorithms were tested across three distinct problem categories: binary (including OneMax, FlipFlop, FourPeaks, SixPeaks, and ContinuousPeaks), permutation (Traveling Salesman and N-Queens), and combinatorial (Knapsack). The performance and behavior of each algorithm were evaluated through multiple experimental trials. Results showed that while MIMIC and GA demonstrated superior performance, they also came with significantly higher computational costs. Ongoing research aims to develop more efficient algorithms that maintain high performance while reducing computational overhead

\section*{Funding}
This research is supported by the Natural Sciences and Engineering Research Council of Canada (NSERC) Collaborative Research and Training Experience (CREATE) program, grant number: 565429-2022.

\section*{Data Availability Statement}
The data generated during the course of this study are available on GitHub at \url{https://github.com/semtu/On-Automated-Object-Grasping-for-Intelligent-Prosthetic-Hands-Using-Machine-Learning}.

\section*{Conflicts of Interest}
The authors declare no conflicts of interest.
